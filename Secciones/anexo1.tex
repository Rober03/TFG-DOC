\section{Anexo I: Preparación del entorno de desarrollo.}

En este anexo se resumen los pasos a seguir para la puesta a punto del entorno de desarrollo, también se incluye la forma de ejecutar el flujo de trabajo y abrir la vista interactiva con el informe de resultados. En primer lugar, se tiene que instalar el entorno de desarrollo Knime, la aplicación está disponible para sistemas operativos Windows, MacOs y Linux. Se puede descargar a través de la siguiente url: \url{https://www.knime.com/downloads}

\bigbreak

Una vez instalado el entorno de desarrollo, se tiene que abrir el fichero EVAL\_MOD\_PRED.knwf incluido en el material. Es posible que en este punto se pida instalar alguna extensión, en este caso se recomienda la instalación por defecto que ofrece Knime. Por último, es necesario guardar el flujo de trabajo en una ruta local.

\bigbreak

Para que funcione el proceso automático es importante crear un directorio en la misma ruta donde previamente se ha guardado el flujo de trabajo. El nombre del directorio tiene que ser DATA, hay que tener en cuenta que el nombre ha de ir en mayúsculas. Otro punto importante a la hora de ejecutar el proceso automático es cambiar el nombre de la clase en la cabecera de los ficheros de datos, por defecto el nombre es CLASS. En el nodo CLASS NAME se puede modificar el nombre que se usa por defecto para la clase, este nodo situado en el flujo principal. Para finalizar, se tienen que incluir los ficheros en el directorio DATA.

\bigbreak

\begin{figure}[htp]
    \centering
    \includegraphics[scale=0.3]{config.png}
    \caption{Flujo principal, se señalan en rojo los componentes con los que tiene que interaccionar el usuario.}
    \label{fig:4}
\end{figure}

\bigbreak

Para la ejecución completa del flujo de trabajo se tiene que hacer clic en el botón verde situado en la barra de herramientas. Alternativamente se puede utilizar el comando shift+F7 para ejecutar el flujo completo. Los resultados se presentan en el nodo REPORT situado en el flujo principal, se puede acceder al informe haciendo clic derecho sobre el nodo y seleccionando la opción de vista interactiva.

\clearpage