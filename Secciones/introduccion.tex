\section{Introducción}

El proyecto consiste en el estudio de las principales técnicas tanto analíticas como gráficas que se utilizan en la actualidad para la evaluación de modelos predictivos. A continuación, se describen los principales objetivos del proyecto:
\begin{itemize}
    \item Introducir y dar a conocer las propiedades y usos de la curva Brier, el objetivo es sentar las bases para futuras investigaciones. 
    \item Presentar algunas de las métricas más usadas en el ámbito del análisis de datos para la evaluación de modelos predictivos. Estas métricas han sido ampliamente documentadas en estudios anteriores, en este documento se propone una revisión de las más destacadas.
    \item Realizar un estudio comparativo de las curvas ROC, PR y Brier. La curva Brier propone un método gráfico para la evaluación de la calidad del modelo similar al propuesto anteriormente en las curvas ROC y PR, en este punto se busca ofrecer una visión general de cada una de las curvas cubriendo aspectos como la definición, el funcionamiento, las propiedades, etc. Por último, presentar una comparativa entre las diferentes curvas y determinar cuales son las ventajas e inconvenientes de la curva Brier frente al resto.
    \item Elaborar una experimentación en la que se definan una serie de casos prácticos con diferentes conjuntos de datos. Para cada caso práctico se deben de aplicar los métodos de evaluación que se presentan en este documento.
\end{itemize}

El contenido del proyecto se divide en dos grandes bloques.
En el primer bloque se realiza el estudio teórico en el que se introduce cada una de las métricas y de los gráficos.
El segundo bloque incluye toda la parte de experimentación, en este bloque se recopilan e interpretan los resultados que se han obtenido en los diferentes casos de prácticos.


\clearpage