\section{Introducción}

El proyecto consiste en el estudio de las principales técnicas tanto analíticas como gráficas que se utilizan en la actualidad para la evaluación de modelos predictivos. A continuación, se describen los principales objetivos del proyecto:

\bigbreak

\begin{itemize}
    \item Introducir y dar a conocer las propiedades y usos de la curva-A. El objetivo es sentar una base sólida para la aplicación de este nuevo método.
    \item Presentar una selección de los métodos más usados para la evaluación de modelos predictivos. Estos métodos han sido ampliamente documentados en estudios anteriores, en este documento se propone una revisión de los más destacados.
    \item Realizar un estudio comparativo entre los diferentes métodos gráficos que ayude a determinar cuáles son las ventajas e inconvenientes de la aplicación de la curva-A. 
    \item Elaborar una experimentación que permita aplicar los métodos de evaluación a diferentes casos prácticos. Cada caso práctico cuenta con un conjunto de datos asociado, los conjuntos de datos cubren diferentes tipos de modelos predictivos.
\end{itemize}

\bigbreak

El contenido del proyecto se divide en dos grandes bloques. En el primer bloque se realiza un estudio teórico en el que se incluye la revisión de los diferentes métodos de evaluación. El segundo bloque cubre la parte experimental del proyecto, en este bloque se recopilan e interpretan los resultados obtenidos para los diferentes casos prácticos.

\clearpage