\section{Técnicas para la Evaluación de Modelos Predictivos}

%En este punto, se puede hablar de conjunto de test, train etc

En este apartado se realiza una revisión de los principales métodos


Los modelos predictivos en los que se centra esta sección son modelos de clasificación en los que se tiene un número discreto de posibles valores para la clase objetivo, por ejemplo, un conjunto de datos con registros de emails, con una clase que marca cada registro como spam o no spam. El modelo predictivo en este caso está entrenado para predecir un valor u otro de esta clase en función del resto de atributos. En esta sección se ofrece un análisis de las diferentes técnicas tanto analíticas como gráficas para la evaluación de modelos predictivos.

\subsection{Matriz de confusión}

La matriz de confusión es una herramienta que se suele aplicar en las primeras fases del análisis, permite organizar los indicadores de rendimiento obtenidos para un modelo predictivo, en general, ofrece un recuento de los registros en base a la clase real de cada registro y a la predicción que realiza el modelo. La matriz de confusión se define como una matriz cuadrada de $NxN$ elementos donde $N$ es el numero de clases presentes en el modelo, cada fila de la matriz de confusión define la clase real del registro y cada columna define la clase predicha por el modelo o viceversa.

\bigbreak

En la matriz de confusión se definen cuatro indicadores, estos indicadores ofrecen una visión general del rendimiento del modelo y son la base para el calculo de algunas métricas mas avanzadas que se tratan en secciones posteriores de este documento. Los indicadores definidos en la matriz de confusión son:

\begin{itemize}
    \item Verdadero Positivo: Registros de clase real positiva y en los que la predicción se hace correctamente.
    \item Verdadero Negativo: Registros de clase real negativa y en los que la predicción se hace correctamente.
    \item Falso Negativo: Registros de clase real positiva y en los que la predicción se hace incorrectamente.
    \item Falso Positivo: Registros de clase real negativa y en los que la predicción se hace incorrectamente.
\end{itemize}

\pagebreak

La matriz de confusión se define de forma general para modelos de clasificación multi-etiqueta, sin embargo, a la hora de obtener los indicadores de rendimiento para modelos de clasificación con mas de dos clases tenemos que definir $N$ matrices una por cada clase del modelo, en cada matriz se define positiva una de las clases y se agrupan como negativas el resto, de esta forma, el estudio que se realiza a partir de las diferentes sub-matrices ya no es un estudio global, sino que es especifico para cada clase concreta.

\begin{table}[ht]
    \centering
    \begin{tabularx}{200pt}{Y Y Y}
                 & Positivo*    & Negativo* \\\hline
        Positivo & TP           & FN        \\\hline
        Negativo & FP           & TN        \\\hline
    \end{tabularx}

    \caption{Matriz de confusión 2x2.}
    \label{tab:1}
\end{table}

%%%%%%%%%%%%%%%%%%%%%%%%%%%%%%%%%%%%%%%%%%%%%%%%%%%%%%%%% SUBSECCION METODOS ESTADISTICOS

\subsection{Métodos estadísticos}

%%%%%%%%%%%%%%%%%%%%%%%%%%%%% SUBSUBSECCION Exactitud

\subsubsection{Exactitud}

La exactitud es una de las medidas más utilizadas a la hora de establecer la calidad general de un modelo predictivo. Este método representa la tasa de acierto que se obtiene al aplicar un modelo predictivo sobre un conjunto de \textit{test}. El calculo de esta medida se hace en base a los cuatro indicadores de rendimiento definidos en la matriz de confusión. El resultado que ofrece se encuentra en el intervalo de cero a uno, los valores próximos a cero corresponden a modelos con una baja tasa de acierto, mientras que los valores próximos a uno corresponden a modelos con una alta tasa de acierto. En términos generales los modelos que obtienen un resultado inferior a media unidad son poco prometedores, en promedio tienen una tasa de aciertos inferior a la de un modelo de clasificación aleatoria. Por último, cabe destacar que es un método independiente del número de clases, se puede aplicar tanto a modelos de clasificación binaria como a modelos de clasificación multi-etiqueta.

\bigbreak

\begin{equation}\tag*{}
    ACC = \frac{TP+TN}{TP+TN+FP+FN}
\end{equation}

%%%%%%%%%%%%%%%%%%%%%%%%%%%%% SUBSUBSECCION SENSIBILIDAD Y ESPECIFICIDAD

\subsubsection{Sensibilidad y Especificidad}

La sensibilidad y la especificidad son dos métodos complementarios que ofrecen una representación de la exactitud del modelo separando los registros en base a la clase real. La sensibilidad es un método que mide la tasa de acierto del modelo sobre instancias de clase positiva. De igual forma, la especificidad mide la tasa de acierto del modelo sobre instancias de clase negativa. En ambos casos el calculo se hace a partir de los indicadores de rendimiento definidos en la matriz de confusión. El resultado obtenido se encuentra en el intervalo de cero a uno, los valores próximos a cero indican una baja tasa de acierto, mientras que los valores próximos a uno indican una alta tasa de acierto. A diferencia de la exactitud estos dos métodos son sensibles al numero de clases del modelo, solo se pueden aplicar a modelos de clasificación binaria. Para su aplicación a modelos de clasificación multi-etiqueta se puede aplicar el método a una clase concreta que definimos como positiva, mientras que el resto de clases se agrupan como una única clase que denominamos negativa.

\bigbreak
\begin{equation}\tag*{}
    TPR = \frac{TP}{P} = \frac{TP}{TP+FN}
    \hspace{1cm}
    TPN = \frac{TN}{N} = \frac{TN}{TN+FP}
\end{equation}


%%%%%%%%%%%%%%%%%%%%%%%%%%%%% SUBSUBSECCION PRECISION Y PRECISION INVERSA

\subsubsection{Precisión y Precisión Inversa}

La precisión y la precisión inversa son métodos que calculan la exactitud del modelo en base a la predicción que se realiza. La precisión es un método que mide la tasa de aciertos del modelo sobre instancias que se predicen positivas. De la misma forma, la precisión inversa se define como la tasa de aciertos del modelo sobre instancias que se predicen negativas. De nuevo, el calculo se hace a partir de los indicadores de rendimiento definidos en la matriz de confusión. También, son métodos sensibles al numero de clases que se definen en el modelo; de hecho, solo se pueden aplicar a modelos de clasificación binaria. Para su aplicación a modelos de clasificación multi-etiqueta es necesario definir una clase en especifico como positiva, mientras que el resto de clases se agrupan como clase clase negativa.

\bigbreak
\begin{equation}\tag*{}
    PPV = \frac{TP}{P^{*}} = \frac{TP}{TP+FP}
    \hspace{1cm}
    NPV = \frac{TN}{N^{*}} = \frac{TN}{TN+FN}
\end{equation}

%%%%%%%%%%%%%%%%%%%%%%%%%%%%% SUBSUBSECCION RAZON DE VEROSIMILITUD

\subsubsection{Razones de Verosimilitud}

Las razones de verosimilitud son métodos que se utilizan fundamentalmente en el ámbito medico para la evaluación de pruebas diagnosticas. Los indicadores miden como varia la probabilidad de padecer una determinada patología en en función de si el paciente presenta o no presenta un estado concreto. De forma similar, para el ámbito de la evaluación de modelos predictivos las razones de verosimilitud nos informan de como varia la probabilidad de que un registro sea positivo en función de la predicción que realiza el modelo.

\bigbreak

Tenemos dos tipos de razones de verosimilitud, la razón de verosimilitud positiva LR+ y la razón de verosimilitud negativa LR-. La razón de verosimilitud positiva mide como varia la probabilidad de que un registro sea positivo cuando se clasifica como positivo. Por otro lado, la razón de verosimilitud negativa mide como varia la probabilidad de que un registro sea positivo cuando se clasifica como negativo. En ambos casos el calculo de los métodos se hace a partir de la sensibilidad y la especificidad. Es importante destacar que las razones de verosimilitud no guardan una relación de proporcionalidad, el cambio en la probabilidad de que un registro sea de clase positiva no varia en la misma proporción cuando el registro se clasifica como positivo que cuando se clasifica como negativo.

\bigbreak

\begin{equation}\tag*{}
    LR^{\phantom{.}+}    = \frac{TPR}{1-TNR}
    \hspace{1cm}
    LR^{\phantom{.}-} = \frac{1-TPR}{TNR}
\end{equation}

\bigbreak

La razón de verosimilitud positiva es una medida que toma valores mayores que la unidad, ofrece un mejor indicador cuanto mayor sea el valor que toma. Por el contrario, la razón de verosimilitud negativa se encuentra en el rango de valores de cero a uno, ofrece un mejor indicador cuanto mas próximo al cero se encuentre. En ambos casos los valores próximos a la unidad indican que el modelo no tiene capacidad discriminatoria, es decir, la predicción que se realiza no varia la probabilidad de pertenecer a la clase positiva. En la Tabla \ref{tab:2} podemos ver una aproximación de la variación que se produce en la probabilidad de pertenecer a la clase positiva para diferentes razones de verosimilitud.

\bigbreak

\begin{table}[ht]
    \centering
    \begin{tabularx}{195pt}{c c c}
        Variación (\%) & \hspace{20pt}LR+\hspace{20pt} & LR- \\\hline
        +45                              & 10                            & -   \\\hline
        +30                              & 5                             & -   \\\hline
        +15                              & 2                             & -   \\\hline
        0                                & 1                             & 1   \\\hline
        -15                              & -                             & 0.5 \\\hline
        -30                              & -                             & 0.2 \\\hline
        -45                              & -                             & 0.1 \\\hline
    \end{tabularx}

    \caption{Variación en la probabilidad de pertenecer a la clase positiva para diferentes razones de verosimilitud \cite{McGee2002}.}
    \label{tab:2}
\end{table}

\bigbreak

En la Tabla \ref{tab:3} tenemos los resultados obtenidos para el fichero de datos breast\_gse26910.csv incluido en el apartado de experimentación. La clase Normal presenta un LR+ de 4 unidades, esto implica un aumento moderado (entre un 15\% y un 30\%) de la probabilidad que tiene un registro de pertenecer a la clase positiva cuando el modelo predice positivo el registro. Por otro lado, la clase Normal presenta un LR- de 0.4 unidades esto supone de nuevo una disminución moderada (entre un 15\% y un 30\%) de la probabilidad que tiene un registro de pertenecer a la clase positiva cuando el modelo predice negativo el registro.

\bigbreak

\begin{table}[ht]
    \centering
    \begin{tabular}[t]{lcc}
            & \hspace{20pt}Normal\hspace{20pt} & Tumoral \\\hline
        TPR & 0.667                            & 0.833   \\\hline
        TNR & 0.833                            & 0.667   \\\hline
        LR+ & 4.000                            & 2.500   \\\hline
        LR- & 0.400                            & 0.250   \\\hline
    \end{tabular}
    \caption{LRs obtenidas a partir del fichero de datos breast\_gse26910.csv}
    \label{tab:3}
\end{table}

%%%%%%%%%%%%%%%%%%%%%%%%%%%%% SUBSUBSECCION DOR

\subsubsection{DOR}

El DOR es un método que surge a partir de la necesidad de tener un solo indicador de rendimiento que sea fácil de interpretar. Este método no depende del balance entre el número de registros positivos y negativos, en otros métodos como por ejemplo la exactitud esta cuestión presenta un problema la hora de obtener resultados concluyentes. El DOR se calcula a partir del las razones de verosimilitud, se define como el cociente de la razón de verosimilitud positiva entre la razón de verosimilitud negativa. El resultado que ofrece es un valor mayor que cero, cuanto mayor sea este valor mejor rendimiento presentará el modelo.

\bigbreak

\begin{equation}\tag*{}
    DOR = \frac{LR^{\phantom{.}+}}{LR^{\phantom{.}-}}
\end{equation}

\bigbreak

En la Tabla \ref{tab:4} tenemos los resultados obtenidos para el fichero de datos breast\_gse26910.csv incluido en el apartado de experimentación. La clase Tumoral presenta un DOR de 10 unidades, esto implica que para un registro que se predice Tumoral la probabilidad de ser Tumoral es 10 veces mayor que la de ser Normal. De la misma forma, para la clase Normal tenemos un DOR de 10 unidades, de nuevo esto indica que un registro que se predice Normal tiene 10 veces mas probabilidad de ser Normal que de ser Tumoral.

\bigbreak

\begin{table}
    \centering
    \begin{tabular}{lcc}
            & \hspace{20pt}Normal\hspace{20pt}  & Tumoral   \\\hline
        LR+ & 4.000                             & 2.500     \\\hline
        LR- & 0.400                             & 0.250     \\\hline
        DOR & 10                                & 10        \\\hline
    \end{tabular}
    \caption{DORs obtenidos a partir del fichero de datos breast\_gse26910.csv}
    \label{tab:4}
\end{table}


%%%%%%%%%%%%%%%%%%%%%%%%%%%%% SUBSUBSECCION Índice Youden

\subsubsection{Índice Youden}

