\thispagestyle{empty}
\section*{Resumen}

En este documento se hace una revisión de los métodos mas representativos que existen en la actualidad para la evaluación de modelos predictivos.
También se introduce el concepto de curva-A, un nuevo gráfico que parte de la fórmula de validación descrita por Brier en el artículo \textit{Verification of Forecasts Expressed in Terms of Probability} \cite{brie_1950} publicado en 1950. La curva-A ofrece una nueva forma de representación gráfica que permite evaluar la calidad de un modelo predictivo, es un método que se puede aplicar tanto a modelos de clasificación binaria cómo a modelos de clasificación multi etiqueta. En el documento se presenta un estudio comparativo entre las curvas ROC y A, el objetivo del estudio es ofrecer una visión general de cada una de las curvas, se cubren aspectos como la definición, el funcionamiento, las propiedades, etc. El estudio en última instancia busca presentar una comparativa entre las dos curvas analizando las ventajas e inconvenientes que ofrece la curva-A.

\bigbreak

En el apartado tecnológico se ha realizado una implementación en KNIME de cada uno de los métodos que se presentan en este documento. También se ha elaborado un flujo de trabajo que permite ejecutar el proceso completo de entrenamiento y evaluación de forma automática para varios conjuntos de datos. El proceso incluye una vista en la que se presenta un informe con los resultados obtenidos utilizando cada conjunto de datos. Finalmente, se incluyen varios casos prácticos en los que se realiza un estudio de la calidad general del modelo en base a los métodos de evaluación implementados.


\clearpage