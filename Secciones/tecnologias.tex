\section{Tecnologías}

En esta sección se cubren las diferentes tecnologías que se han utilizado para el desarrollo de este trabajo final de grado, también se detalla el uso que se  le ha dado a cada una de las herramientas. 

\bigbreak

En primer lugar tenemos Knime, Knime es una potente herramienta de código abierto orientada al análisis de datos, es ideal para proyectos relacionados con minería de datos, inteligencia artificial, etc. Knime es la herramienta principal con la que se ha desarrollado la parte experimental del proyecto, el objetivo principal para uso de Knime es ofrecer una implementación de los diferentes métodos y gráficas que se han ido desarrollando a lo largo del documento. En este sentido, se realiza un flujo de trabajo completo en el que se incluyen las fases de carga de datos, de generación del modelo predictivo, de evaluación del modelo y finalmente de reporte de resultados. Cabe destacar que se han utilizado algunas extensiones que aumentan la funcionalidad de la herramienta, como es el caso de la extensión para la integración de Python o la extensión para la generación de datos. A continuación, se enumeran algunas de las principales ventajas que ofrece Knime:

\begin{itemize}
    \item Permite la creación de un flujo de trabajo que incluye las principales fases que componen un proyecto de análisis de datos. 
    \item Knime ofrece una interfaz visual e intuitiva que permite a través del uso de nodos la visualización de datos, la creación de modelos predictivos, la generación de reportes, etc.
    \item Es una herramienta de código abierto.
    \item Permite la integración con lenguajes de programación como Java, Python, R, etc. Esto aporta un extra de flexibilidad a la hora de hacer ciertas operaciones o utilizar librerías propias de un lenguaje especifico.
    \item La aplicación de escritorio esta disponible para sistemas operativos Windows, MacOs y Linux.
\end{itemize}

\bigbreak

Otra de las aplicaciones que más utilizadas a lo largo del desarrollo ha sido Git. Git es un sistema de control de versiones distribuido, permite mantener un control sobre el desarrollo que se ha realizado. Git junto con Github me han permitido tener una copia de seguridad actualizada en la nube, también me han habilitado la posibilidad de desarrollar el proyecto desde varios equipos de forma sencilla.

\clearpage