\section{Experimentación}

\subsection{Descripción de la Experimentación}

En esta sección se cubre la parte experimental del proyecto. Todo el desarrollo ha sido realizado utilizando Knime, una herramienta para el análisis de datos ampliamente utilizada en este sector. El objetivo principal es ofrecer una comparación entre las diferentes métricas vistas en secciones anteriores, poniendo especial énfasis en la comparación de estas con el nuevo método grafico curva-A.

\bigbreak

Para la experimentación vamos a partir de una selección de seis ficheros de datos, todos los ficheros utilizados en la experimentación han sido obtenidos de la web del SBCB \url{https://sbcb.inf.ufrgs.br/cumida}, en general los conjuntos de datos forman parte de un repositorio de datos relacionado con diversos tipos de cáncer, estos conjuntos de datos cubren un amplio rango de casuísticas y suponen una excelente elección para entrenar modelos tanto de clasificación binaria como modelos de clasificación multi etiqueta.

\bigbreak

El escenario de pruebas que se ha definido no se realiza ningún tipo de preprocesado al conjunto de datos de entrada, únicamente se aplica una normalización, ya que supone una mejora en el rendimiento de los modelos basados en el teorema de Bayes. Para los diferentes conjuntos de datos, se va a aplicar un predictor basado en el teorema de Bayes. Para la validación aplicamos validación cruzada con 10 iteraciones, el uso de este tipo de validación está muy extendido en proyectos de análisis de datos, ya que ofrece en general grandes resultados. Para cada fichero presentaremos los resultados obtenidos fruto de aplicar las métricas vistas anteriormente al conjunto de datos correspondiente. Posteriormente se exponen las conclusiones.


\clearpage

\subsection{Resultados de la Experimentación}


%%%%%%%%%%%%%%%%%%%%%%%%%%%%%%%%%%%%%%%%%%%%%%%%%%%%%%% RESULTADO
\subsubsection{Fichero brain\_gse15824.csv}

\begin{table}[htp]
    \small
    \centering
    \begin{tabularx}{\columnwidth}{Y Y}
        ACC       & AUAC    \\\hline
        $0.757$   & $0.764$ \\\hline
    \end{tabularx}
    \caption{Resultados globales para el fichero brain\_gse15824.csv.}
    \label{tab:10}
\end{table}

\begin{table}[htp]
    \small
    \centering
    \begin{tabularx}{\columnwidth}{l c c c c}
                &  Astrocytoma  & Glioblastoma & Oligodendrioglioma   & Glioblastoma-cell-line   \\\hline
        TPR     &  $0.500$      & $1.000$      & $0.286$              & $1.000$                  \\\hline
        TNR     &  $0.966$      & $0.720$      & $0.967$              & $1.000$                  \\\hline
        PPV     &  $0.800$      & $0.632$      & $0.667$              & $1.000$                  \\\hline
        NPV     &  $0.875$      & $1.000$      & $0.853$              & $1.000$                  \\\hline
        LR+     &  $14.500$     & $3.571$      & $8.571$              & -                        \\\hline
        LR-     &  $0.518$      & $0.000$      & $0.739$              & $0.000$                  \\\hline
        DOR     &  $28.000$     & -            & $11.600$             & -                        \\\hline
        YI      &  $0.466$      & $0.720$      & $0.252$              & $1.000$                  \\\hline
        MCC     &  $0.561$      & $0.674$      & $0.362$              & $1.000$                  \\\hline
        DP      &  $0.798$      & -            & $0.587$              & -                        \\\hline
        $F_{1}$ &  $0.615$      & $0.774$      & $0.400$              & $1.000$                  \\\hline
        MK      &  $0.675$      & $0.632$      & $0.520$              & $1.000$                  \\\hline
        BCR     &  $0.733$      & $0.860$      & $0.626$              & $1.000$                  \\\hline
        GM      &  $0.695$      & $0.849$      & $0.526$              & $1.000$                  \\\hline
        OP      &  $0.547$      & $0.648$      & $0.294$              & $1.000$                  \\\hline
        Jaccard &  $0.444$      & $0.632$      & $0.250$              & $1.000$                  \\\hline

    \end{tabularx}
    \caption{Resultados agrupados por clase para el fichero brain\_gse15824.csv.}
    \label{tab:11}
\end{table}

\bigbreak

\begin{figure}[htp]
    \centering
     \subfloat[Curva ROC]{
       \includegraphics[width=0.4\textwidth]{brain_gse15824_ROC.PNG}}
     \subfloat[Curva A]{
       \includegraphics[width=0.4\textwidth]{brain_gse15824_A_CURVE.PNG}}
    \caption{Curvas obtenidas para el fichero brain\_gse15824.csv.}
    \label{fig:10}
\end{figure}

\bigbreak

El fichero de datos brain\_gse15824.csv presenta una clase con cuatro etiquetas, las etiquetas son Astrocytoma, Glioblastoma, Oligodendrioglioma y Glioblastoma-cell-line. En la siguietne lista se ofrece una interpretación de los valores obtenidos al aplicar los métodos de evaluación vistos en secciones anteriores.

\bigbreak

\begin{itemize}
    \item La exactitud es de 0.757 unidades, es decir, hay un $75.7$\% de probablidad de acierto en la predicción.
    \item El área bajo la curva-A es de $0.764$ un valor notablemente alto. La diferencia entre el área bajo la curva-A y la exactitud es de $0.007$ unidades.
    \item La sensibilidad para las clases Glioblastoma y Glioblastoma-cell-line es perfecta, todas las instancias de estas clases se predicen de forma correcta. La clase Astrocytoma tiene una sensibilidad de $0.5$ unidades, tan solo la mitad de las instancias de está clase se predicen de forma correcta. La etiqueta Oligodendrioglioma presenta una sensibilidad de $0.286$ unidades, el $28.6$\% de las instancias de esta clase se predicen correctamente.
    \item La clase Glioblastoma-cell-line presenta una especificidad perfecta, ninguna instancia de otra clase se predice de clase Glioblastoma-cell-line. El resto de clases presentan tasas de especificidad notablemente buenas, en el caso de las clases Astrocytoma y Oligodendrioglioma presentan un valor muy similar de aproximadamente $0.96$ unidades. El indicador más bajo de especificidad lo presenta la clase Glioblastoma con un valor de $0.72$ que indica una tasa alta de probabilidad de acierto al determinar que una instancia no pertenece a esta clase.
    \item El indicador de precisión obtenido para la clase Glioblastoma-cell-line es de una unidad, todas las predicciónes que establecen que una instancia pertenece a esta clase son correctas. La clase Astrocytoma tiene un valor de precision de $0.8$ unidades, el 80\% de prediciones que se realizan sobre esta clase son correctas. Las clases Glioblastoma y Oligodendrioglioma presentan tasas de precisión muy similares ambas se encuentran situan entorno a $0.65$ unidades, en general son tasas aceptables.
    \item Las etiquetas Glioblastoma y Glioblastoma-cell-line obtienen un valor de precisión inversa de una unidad, todas las instancias que no se predicen de estas clases, no son en efecto instancias de estas clases. Las etiquetas d
\end{itemize}

\clearpage

%%%%%%%%%%%%%%%%%%%%%%%%%%%%%%%%%%%%%%%%%%%%%%%%%%%%%%% RESULTADO

\subsubsection{Fichero breast\_gse45827.csv}

\begin{table}[htp]
    \small
    \centering
    \begin{tabularx}{\columnwidth}{Y Y}
        ACC       & AUAC    \\\hline
        $0.914$   & $0.917$ \\\hline
    \end{tabularx}
    \caption{Resultados globales para el fichero breast\_gse45827.csv.}
    \label{tab:12}
\end{table}

\begin{table}[htp]
    \small
    \centering
    \begin{tabularx}{\columnwidth}{l Y Y Y Y Y Y}
                &  Her          & Basal     & Cell\_line & Luminal\_a & Luminal\_b & Normal    \\\hline
        TPR     &  $0.900$      & $0.927$   & $1.000$    & $0.862$    & $0.933$    & $0.857$   \\\hline
        TNR     &  $0.967$      & $0.982$   & $1.000$    & $0.984$    & $0.959$    & $1.000$   \\\hline
        PPV     &  $0.871$      & $0.950$   & $1.000$    & $0.926$    & $0.848$    & $1.000$   \\\hline
        NPV     &  $0.975$      & $0.973$   & $1.000$    & $0.968$    & $0.983$    & $0.993$   \\\hline
        LR+     &  $27.225$     & $50.976$  & -          & $52.586$   & $22.587$   & -         \\\hline
        LR-     &  $0.103$      & $0.075$   & $0.000$    & $0.140$    & $0.070$    & $0.143$   \\\hline
        DOR     &  $263.250$    & $684.000$ & -          & $375.000$  & $324.800$  & -         \\\hline
        YI      &  $0.867$      & $0.909$   & $1.000$    & $0.846$    & $0.892$    & $0.857$   \\\hline
        MCC     &  $0.856$      & $0.916$   & $1.000$    & $0.869$    & $0.861$    & $0.923$   \\\hline
        DP      &  $1.334$      & $1.563$   & -          & $1.419$    & $1.385$    & -         \\\hline
        $F_{1}$ &  $0.885$      & $0.938$   & $1.000$    & $0.893$    & $0.889$    & $0.923$   \\\hline
        MK      &  $0.846$      & $0.923$   & $1.000$    & $0.894$    & $0.832$    & $0.993$   \\\hline
        BCR     &  $0.933$      & $0.954$   & $1.000$    & $0.923$    & $0.946$    & $0.929$   \\\hline
        GM      &  $0.933$      & $0.954$   & $1.000$    & $0.921$    & $0.946$    & $0.926$   \\\hline
        OP      &  $0.918$      & $0.938$   & $1.000$    & $0.894$    & $0.940$    & $0.916$   \\\hline
        Jaccard &  $0.794$      & $0.884$   & $1.000$    & $0.806$    & $0.800$    & $0.857$   \\\hline
    \end{tabularx}
    \caption{Resultados agrupados por clase para el fichero breast\_gse45827.csv.}
    \label{tab:13}
\end{table}

\clearpage

\begin{figure}[htp]
    \centering
     \subfloat[Curva ROC]{
       \includegraphics[width=0.4\textwidth]{breast_gse45827_ROC.PNG}}
     \subfloat[Curva A]{
       \includegraphics[width=0.4\textwidth]{breast_gse45827_A_CURVE.PNG}}
    \caption{Curvas obtenidas para el fichero breast\_gse45827.csv.}
    \label{fig:11}
\end{figure}

\bigbreak

\lipsum[1]

\clearpage

%%%%%%%%%%%%%%%%%%%%%%%%%%%%%%%%%%%%%%%%%%%%%%%%%%%%%%% RESULTADO

\subsubsection{Fichero colorectal\_gse21510.csv}

\begin{table}[htp]
    \small
    \centering
    \begin{tabularx}{\columnwidth}{Y Y}
        ACC       & AUAC    \\\hline
        $0.980$   & $0.983$ \\\hline
    \end{tabularx}
    \caption{Resultados globales para el fichero colorectal\_gse21510.csv.}
    \label{tab:14}
\end{table}

\begin{table}[htp]
    \small
    \centering
    \begin{tabularx}{\columnwidth}{l Y Y Y}
                &  Normal\_homogenized  & Tumoral\_lcm  & Tumoral\_homogenized  \\\hline
        TPR     &  $0.920$              & $1.000$       & $0.944$               \\\hline
        TNR     &  $1.000$              & $0.930$       & $1.000$               \\\hline
        PPV     &  $1.000$              & $0.972$       & $1.000$               \\\hline
        NPV     &  $0.984$              & $1.000$       & $0.992$               \\\hline
        LR+     &  -                    & $14.333$      & -                     \\\hline
        LR-     &  $0.080$              & $0.000$       & $0.056$               \\\hline
        DOR     &  -                    & -             & -                     \\\hline
        YI      &  $0.920$              & $0.930$       & $0.944$               \\\hline
        MCC     &  $0.951$              & $0.951$       & $0.968$               \\\hline
        DP      &  -                    & -             & -                     \\\hline
        $F_{1}$ &  $0.958$              & $0.986$       & $0.971$               \\\hline
        MK      &  $0.984$              & $0.972$       & $0.992$               \\\hline
        BCR     &  $0.960$              & $0.965$       & $0.972$               \\\hline
        GM      &  $0.959$              & $0.964$       & $0.972$               \\\hline
        OP      &  $0.945$              & $0.943$       & $0.965$               \\\hline
        Jaccard &  $0.920$              & $0.972$       & $0.944$               \\\hline
    \end{tabularx}
    \caption{Resultados agrupados por clase para el fichero colorectal\_gse21510.csv.}
    \label{tab:15}
\end{table}

\clearpage

\begin{figure}[htp]
    \centering
     \subfloat[Curva ROC]{
       \includegraphics[width=0.4\textwidth]{colorectal_gse21510_ROC.PNG}}
     \subfloat[Curva A]{
       \includegraphics[width=0.4\textwidth]{colorectal_gse21510_A_CURVE.PNG}}
    \caption{Curvas obtenidas para el fichero colorectal\_gse21510.csv.}
    \label{fig:12}
\end{figure}

\bigbreak

\lipsum[1]

\clearpage


%%%%%%%%%%%%%%%%%%%%%%%%%%%%%%%%%%%%%%%%%%%%%%%%%%%%%%% RESULTADO

\subsubsection{Fichero breast\_gse42568.csv}

\begin{table}[htp]
    \small
    \centering
    \begin{tabularx}{\columnwidth}{Y Y}
        ACC       & AUAC    \\\hline
        $0.991$   & $0.996$ \\\hline
    \end{tabularx}
    \caption{Resultados globales para el fichero breast\_gse42568.csv.}
    \label{tab:16}
\end{table}

\begin{table}[htp]
    \small
    \centering
    \begin{tabularx}{\columnwidth}{l Y Y}
                &  Normal               & Tumoral       \\\hline
        TPR     &  $0.933$              & $1.000$       \\\hline
        TNR     &  $1.000$              & $0.933$       \\\hline
        PPV     &  $1.000$              & $0.990$       \\\hline
        NPV     &  $0.990$              & $1.000$       \\\hline
        LR+     &  -                    & $15.000$      \\\hline
        LR-     &  $0.067$              & $0.000$       \\\hline
        DOR     &  -                    & -             \\\hline
        YI      &  $0.933$              & $0.933$       \\\hline
        MCC     &  $0.961$              & $0.961$       \\\hline
        DP      &  -                    & -             \\\hline
        $F_{1}$ &  $0.966$              & $0.995$       \\\hline
        MK      &  $0.990$              & $0.990$       \\\hline
        BCR     &  $0.967$              & $0.967$       \\\hline
        GM      &  $0.966$              & $0.966$       \\\hline
        OP      &  $0.957$              & $0.957$       \\\hline
        Jaccard &  $0.933$              & $0.990$       \\\hline
    \end{tabularx}
    \caption{Resultados agrupados por clase para el fichero breast\_gse42568.csv.}
    \label{tab:17}
\end{table}

\clearpage

\begin{figure}[htp]
    \centering
     \subfloat[Curva ROC]{
       \includegraphics[width=0.4\textwidth]{breast_gse42568_ROC.PNG}}
     \subfloat[Curva A]{
       \includegraphics[width=0.4\textwidth]{breast_gse42568_A_CURVE.PNG}}
    \caption{Curvas obtenidas para el fichero breast\_gse42568.csv.}
    \label{fig:13}
\end{figure}

\bigbreak

\lipsum[1]

\clearpage


%%%%%%%%%%%%%%%%%%%%%%%%%%%%%%%%%%%%%%%%%%%%%%%%%%%%%%% RESULTADO

\subsubsection{Fichero gastric\_gse79973.csv}

\begin{table}[htp]
    \small
    \centering
    \begin{tabularx}{\columnwidth}{Y Y}
        ACC       & AUAC    \\\hline
        $0.900$   & $0.921$ \\\hline
    \end{tabularx}
    \caption{Resultados globales para el fichero gastric\_gse79973.csv.}
    \label{tab:18}
\end{table}

\begin{table}[htp]
    \small
    \centering
    \begin{tabularx}{\columnwidth}{l Y Y}
                &  Adenocarcinoma       & Normal        \\\hline
        TPR     &  $0.900$              & $0.900$       \\\hline
        TNR     &  $0.900$              & $0.900$       \\\hline
        PPV     &  $0.900$              & $0.900$       \\\hline
        NPV     &  $0.900$              & $0.900$       \\\hline
        LR+     &  $9.000$              & $9.000$       \\\hline
        LR-     &  $0.111$              & $0.111$       \\\hline
        DOR     &  $81.000$             & $81.000$      \\\hline
        YI      &  $0.800$              & $0.800$       \\\hline
        MCC     &  $0.800$              & $0.800$       \\\hline
        DP      &  $1.052$              & $1.052$       \\\hline
        $F_{1}$ &  $0.900$              & $0.900$       \\\hline
        MK      &  $0.800$              & $0.800$       \\\hline
        BCR     &  $0.900$              & $0.900$       \\\hline
        GM      &  $0.900$              & $0.900$       \\\hline
        OP      &  $0.900$              & $0.900$       \\\hline
        Jaccard &  $0.818$              & $0.818$       \\\hline
    \end{tabularx}
    \caption{Resultados agrupados por clase para el fichero gastric\_gse79973.csv.}
    \label{tab:19}
\end{table}

\bigbreak

\begin{figure}[htp]
    \centering
     \subfloat[Curva ROC]{
       \includegraphics[width=0.4\textwidth]{gastric_gse79973_ROC.PNG}}
     \subfloat[Curva A]{
       \includegraphics[width=0.4\textwidth]{gastric_gse79973_A_CURVE.PNG}}
    \caption{Curvas obtenidas para el fichero gastric\_gse79973.csv.}
    \label{fig:14}
\end{figure}


La predicción que se realiza a partir de este fichero de datos ofrece unos indicadores de calidad simétricos para ambas clases. La exactitud es de $0.900$ puntos, mientras que el área bajo la curva-A es de $0.921$, esto supone un aumento leve del área bajo la curva en $0.21$ puntos. La sensibilidad, la especificidad, la precisión y la precisión inversa ofrecen el mismo resultado $0.900$, esto indica que los registros de ambas clases, así como, la predicción que se realiza sobre las diferente clases tienen una tasa de acierto del 90\%. EL índice de verosimilitud positiva indica un aumento del 40\% en la probabilidad de que la predicción sea correcta. El índice de verosimilitud negativa indica una disminución del 45\% en la probabilidad de que la predicción se haga de forma incorrecta. El valor de $0.8$ en el indice YI establece una buena capacidad de predicción a partir de la sensibilidad y la especificidad. El coeficiente de correlación de Matthews según la Tabla \ref{tab:3} indica un fuerte grado de correlación entre la clase y la predicción. El DP establece que el modelo presenta una capacidad discriminatoria limitada.

La representación gráfica de la curva ROC ofrece una visión en la que 

\clearpage

%%%%%%%%%%%%%%%%%%%%%%%%%%%%%%%%%%%%%%%%%%%%%%%%%%%%%%% RESULTADO

\subsubsection{Fichero leukemia\_gse14317.csv}

\begin{table}[htp]
    \small
    \centering
    \begin{tabularx}{\columnwidth}{Y Y}
        ACC       & AUAC    \\\hline
        $0.920$   & $0.938$ \\\hline
    \end{tabularx}
    \caption{Resultados globales para el fichero leukemia\_gse14317.csv.}
    \label{tab:20}
\end{table}

\begin{table}[htp]
    \small
    \centering
    \begin{tabularx}{\columnwidth}{l Y Y}
                &  Atl                  & Normal        \\\hline
        TPR     &  $1.000$              & $0.714$       \\\hline
        TNR     &  $0.714$              & $1.000$       \\\hline
        PPV     &  $0.900$              & $1.000$       \\\hline
        NPV     &  $1.000$              & $0.900$       \\\hline
        LR+     &  $3.500$              & -             \\\hline
        LR-     &  $0.000$              & $0.286$       \\\hline
        DOR     &  -                    & -             \\\hline
        YI      &  $0.714$              & $0.714$       \\\hline
        MCC     &  $0.802$              & $0.802$       \\\hline
        DP      &  -                    & -             \\\hline
        $F_{1}$ &  $0.947$              & $0.833$       \\\hline
        MK      &  $0.900$              & $0.900$       \\\hline
        BCR     &  $0.857$              & $0.857$       \\\hline
        GM      &  $0.845$              & $0.845$       \\\hline
        OP      &  $0.753$              & $0.753$       \\\hline
        Jaccard &  $0.900$              & $0.714$       \\\hline
    \end{tabularx}
    \caption{Resultados agrupados por clase para el fichero leukemia\_gse14317.csv.}
    \label{tab:21}
\end{table}

\bigbreak

\begin{figure}[htp]
    \centering
     \subfloat[Curva ROC]{
       \includegraphics[width=0.4\textwidth]{leukemia_gse14317_ROC.PNG}}
     \subfloat[Curva A]{
       \includegraphics[width=0.4\textwidth]{leukemia_gse14317_A_CURVE.PNG}}
    \caption{Curvas obtenidas para el fichero leukemia\_gse14317.csv.}
    \label{fig:15}
\end{figure}



La exactitud presenta una tasa de aciertos notablemente buena, de cada 100 registros 92 se predicen de forma correcta. El área bajo la curva-A obtiene un valor muy similar al que ha obtenido la exactitud, la diferencia entre ambos métodos supone un aumento en área bajo la curva-A de $0.18$ puntos sobre la exactitud. La sensibilidad y la especificidad representan buenos indicadores, todos los registros de clase Atl se predicen correctamente, los registros de clase Normal tienen un $71.4$\% de probabilidad de acierto. La precisión y la precisión inversa establecen un $90$\% de probabilidad de acierto cuando se predice un registro Atl, por otro la predicción es siempre correcta cuando se predice Normal un registro. El índice de verosimilitud positiva indica un aumento de entorno al 20\% en la probabilidad que tiene un registro que se predice Atl de ser Atl, la clase Normal no presenta índice de verosimilitud positiva debido a que todos los registros que se precicen Normales son efectivamente de clase Normal. El índice de verosimilitud negativa indica que para registros que se predicen de clase Atl hay una reducción del 30\% en la probabilidad de que sea Normal, por otro lado el modelo siempre acierta cuando predice un registro de clase Normal.

\bigbreak

El índice YI presenta un resultado que indica una tasa de aciertos notablemante positiva. El coeficiente de correlación de Matthews de $0.802$ establece una fuerte correlación entre clase y predicción. La medida-F de $0.947$ y $0.833$ implica una que el modelo tiene alta tasa de sensibilidad sobre precisión. El BCR y la GM indican una alta tasa de acierto en la predicción en relación con la sensibilidad y la especificidad del modelo. La precisión de optimización ofrece un resultado en el que se esta penalizando la diferencia entre sensibilidad y especificidad, en valor de $0.714$ en sensibilidad y especificidad implica una reducción leve de la exactitud, sin embargo, a pesar de la penalización sigue ofreciendo un buen nivel de exactitud. El índice de Jaccard establece un nivel de similitud notable entre clase y predicción.

\bigbreak

La representación gráfica que ofrece la curva ROC indican que el modelo ofrece una buena tasa de acierto sobre error en la predicción de ambas clases. La interpretación gráfica de la curva-A establece que la calidad del modelo predictivo es muy positiva.


\clearpage

