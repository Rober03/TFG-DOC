\section{Conclusiones}

\subsection{Científicas}

Tras finalizar el conjunto de pruebas, se puede ver más allá de los resultados una mejora en la claridad con la que se presenta la información utilizando la curva-A respecto a la curva ROC. La mejora se hace más notable en modelos con varias clases, a medida que aumenta el número de clases se hace más evidente la mejora. Este aumento de la claridad con la que se presenta la información se puede ver en los resultados obtenidos para el fichero de datos breast\_gse45827.csv, con un total de seis etiquetas se hace complicado interpretar la información que ofrece la curva ROC. La curva-A supone también una alternativa interesante en determinados casos en los que las diferencias entre las distintas curvas ROC es notable. Un nivel de diferencia elevado entre las curvas ROC implica un aumento de la complejidad en la interpretación de la calidad del modelo. En los resultados obtenidos para el fichero bladder\_gse40355.csv, se puede observar que las diferencias que existen entre las tres curvas complican la interpretación de la calidad general del modelo.

\bigbreak

Los resultados obtenidos en la experimentación indican que por norma general el valor del área bajo la curva-A es ligeramente superior al del resto de métodos. La curva-A, en concreto el área bajo la curva-A busca obtener un indicador global de la calidad del modelo, el otro método de carácter global que se ha tratado en este documento es la exactitud. La diferencia entre la exactitud y el área bajo la curva-A es inferior al 2\% para todos los casos prácticos. Este grado de similitud entre ambas medidas aporta una base sólida sobre la que considerar el AUAC como un método fiable a la hora de evaluar la calidad de un modelo predictivo.

\subsection{Académicas}

La evaluación de modelos predictivos es una rama que se enmarca dentro del análisis de datos, esta temática se ha tratado en varias asignaturas. Principalmente es objeto de estudio en minería de datos, en esta asignatura se introduce el uso de Knime como herramienta para el análisis de datos, también se tratan algunos conceptos que se incluyen en este documento como la exactitud, la sensibilidad, la curva ROC, etc. Otras asignaturas como inteligencia artificial o inteligencia de negocio también tratan la temática, aunque de forma más superficial.

\bigbreak

En la actualidad, el aumento en la demanda de servicios informáticos por parte empresas y particulares está provocando un aumento en la cantidad de datos que se están generando. El aumento en la cantidad de datos disponibles, está provocando una tendencia hacia modelos de negocio basados en el análisis y en la explotación de datos. En este sentido, el proyecto ofrece la oportunidad profundizar en las bases de la evaluación de modelos predictivos, una fase clave en muchos procesos de explotación de datos. En el apartado tecnológico, la herramienta Knime ofrece un servicio escalable que es ideal para incorporarlo en proyectos a nivel profesional.

\subsection{Trabajo Futuro}

En este documento se recopilan algunos de los principales métodos que se aplican en la evaluación de modelos predictivos. La propia temática que implica un trabajo de investigación está sujeta a mejoras continuas en la profundidad del estudio que se realiza. A continuación, se enumeran algunos puntos de mejora que se pueden realizar en futuros trabajos.

\begin{itemize}
    \item Profundizar en la comparativa entre la curva-A y la curva ROC.
    \item Profundizar en la comparativa entre el área bajo la curva-A y la exactitud.
    \item Aumentar el número de casos prácticos, ampliando la casuística en los conjuntos de datos.
\end{itemize}

\clearpage